\cvsection{Research Experience}
\begin{cventries}
    \cventry
		{Fellow}
		{Insight Data Science}
		{Remote}
		{Jan. 2018 - Present}
		{
			\begin{cvitems}
				\item{Implemented a rating system to score dogs based on the similarity between searched personalities and pet description using NMF topic modeling and word2vec.}
				\item{Trained and validated Na\"{i}ve Bayes classifier to predict whether sentences described pets for adoption.}
				\item{Queried, processed, and stored pet-descriptions from Petfinder's RESTful API using NLTK, spaCy, pandas, and PostgreSQL.}
			    \item{Built interactive front-end web-app to select features in search queries using Bootstrap and Flask.}
			\end{cvitems}
		}

	\cventry
		{Graduate Research Fellow}
		{University of Oregon}
		{Eugene, Oregon}
		{Sep. 2012 - Sep. 2017}
		{
			\begin{cvitems}
				\item{Predicted macroscopic conformational states from 100+ GB DNA simulations using k-means and Ward clustering.}
				\item{Evaluated kinetic mechanisms in DNA un-stacking by modeling inter-state transitions as a Markov chain and analyzing through transition path theory.}
				\item{Estimated relaxation timescales for DNA un-stacking by spectral decomposition of transition likelihoods.}
				\item{Coarse-grained atomistic representation from 1000+ atoms to <10 features using PCA and time-lagged ICA.}
				\item{Designed and implemented ``Gradient Adaptive Decomposition'' algorithm, improving predictive accuracy of kinetic classification by over 20\% using NumPy, SciPy, and Scikit-learn.}
				\item{Facilitated training of incoming undergraduate and graduate students in Bash shell, coding with Python and Fortran, and molecular dynamics simulations.}
			\end{cvitems}
		}

	\cventry
		{Research Assistant}
		{St. Edward's University}
		{Austin, Texas}
		{Jan. 2009 - Jun. 2012}
		{
			\begin{cvitems}
			    \item{Collaborated between synthetic and computational groups in the design of novel inhibitors targeting precursors to Rheumatoid Arthritis.}
			    \item{Validated the efficacy of inhibitors through docking studies to reduce the number of synthesized compounds from 1000+ to 100.}
			\end{cvitems}
		}
\end{cventries}
