%-------------------------------------------------------------------------------
%	SECTION TITLE
%-------------------------------------------------------------------------------
\cvsection{Research Experience}

% \begin{cventries}
%     \cventry
%         {Organization}
%         {Institution}
%         {Location}
%         {Date(s)}
%         {Description}
% \end{cventries}

%-------------------------------------------------------------------------------
%	CONTENT
%-------------------------------------------------------------------------------
\begin{cventries}
  \cventry
    {Marina Guenza Lab}
    {University of Oregon}
    {Eugene, Oregon}
    {Sep. 2012 - Sep. 2017}
    {
      Research efforts focused on modeling the conformational dynamics of DNA unstacking/unwinding mechanism, a key physical process involved in thermally induced breathing fluctuations that allow accessibility in protein recognition. To model the slow barrier crossing inherent in rare event transition, microsecond Molecular Dynamics (MD) simulation were analyzed by Markov State Models (MSM) which build a network of stochastic transitions between discrete states along the coordinates of slow-order parameters.
    }
    
  \cventry
    {Eamonn Healy Lab}
    {St. Edward's University}
    {Austin, Texas}
    {Jan. 2009 - Jun. 2012}
    {
      Structural studies comparing ADAM10 and ADAM17 to study drug potency and selectivity. The ADAM family is a class of enzymes largely linked to inflammatory response in pathologies such as rheumatoid arthritis and metastasis in cancers. Due to the large homology both in sequence and conformation in the active site, drugs targeting ADAM17 typically result in favorable selectivity towards several other members of the ADAM family, leading to side effects.  Docking studies were performed using methods within Accelrys Discovery Studio to structurally evaluate novel acetylenic based inhibitors that targeted structural motifs and the associated dynamics found in ADAM17 but not ADAM10.
    }
\end{cventries}
